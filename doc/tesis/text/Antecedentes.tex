\chapter*{Antecedentes}
En Interfaz Cerebro ? Computadora para el Control de un Cursor Basado en Ondas Cerebrales\cite{ref2} se plantea una interfaz que permita la comunicaci�n entre el usuario y la computadora, haciendo uso de sus ondas cerebrales, para el control de un cursor en pantalla mediante comandos obtenidos de las lecturas de un amplificador de ondas cerebrales.

ROS
ROS (Robot Operating System) provee librer�as y herramientas para ayudar a los desarrolladores de software a crear aplicaciones para robots. ROS provee abstracci�n de hardware, controladores de dispositivos, librer�as, herramientas de visualizaci�n, comunicaci�n por mensajes, administraci�n de paquetes y m�s[4]. ROS es utilizado en el software desarrollado para establecer una comunicaci�n por mensajes en tiempo real. 


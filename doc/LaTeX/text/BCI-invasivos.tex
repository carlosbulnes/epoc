\subsection*{BCI Invasivos}
\markboth{\MakeUppercase{BCI Invasivos}}{}
\addcontentsline{toc}{subsection}{BCI Invasivos}

Los dispositivos BCI invasivos son implantados directamente en el cerebro y tienen la m�s alta calidad de se�ales de los BCI. Estos dispositivos son usados para recuperar funciones de personas con par�lisis. Los BCI invasivos son usados tambi�n para recuperar la vista conectando el cerebro a c�maras externas y para restaurar el uso de extremidades usando brazos y piernas rob�ticos controlados por el cerebro. Como todos los dispositivos que se encuentran instalados en la materia gris del cerebro este tipo de BI produce una calidad de se�ales muy alta pero son propensos a causar cicatrices en el tejido cerebral causando que las se�ales comiencen a volverse d�biles o incluso la p�rdida de las reacciones del cuerpo por contener un objeto desconocido en el cerebro\cite{invasiveBCI1}. \\
En las ciencias de la visi�n los implantes en el cerebro han sido usados para tratar la ceguera no cong�nita\footnote{Una enfermedad cong�nita es aquella que se manifiesta desde el nacimiento, ya sea por un trastorno ocurrido durante el desarrollo embrionario, durante el parto o como consecuencia de un defecto hereditario}. William Dobell es uno de los primeros cient�ficos que vienen trabajando con una interfaz cerebro para restaurar la vista como una investigaci�n privada. El implant� su primer prototipo en Jerry, un hombre que qued� ciego en su adultez en 1978. �l insert� un BCI de 68 electrodos en la corteza visual de Jerry y logr� producir la sensaci�n  de ver una luz. En 2012 el experimento fu� realizado en Jens Neumann donde Dobell utiliz� un implante m�s sofisticado que permiti� un mejor mapeo. Investigadores de la Universidad de Emory en Atlanta, dirigidos por Philip Kennedy y Roy Bakay fueron los primeros en instalar un implante en el cerebro de un ser humano que produce se�ales de alta calidad suficientes para estimular el movimiento\cite{invasiveBCI1}. \\
Thomas Navin Lal et al. en su art�culo \cite{invasiveBCI1} desarrollaron un BCI llamado Thought Translation Device (TTD) es cual usan para ayudar a comunicarse a pacientes con par�lisis los cu�les han perdido sus funciones cognitivas. Para poder usar el TTD los pacientes debieron aprender a regular a voluntad su Slow Cortical Potentials (SCP)\footnote{Se llaman SCP (o potenciales corticales lentos en espa�ol) a los cambios relacionados con los eventos de corriente continua lenta obtenidas con los EEG provenientes de los grandes conjuntos de c�lulas en la capa cortical superior\cite{SCP:Online}.}. El sistema entonces permite a su usuario escribir textos en la pantalla de una computadora o navegar en internet. El sistema sin embargo cuenta con dos desventajas: no todos los pacientes logran controlar su SCP adem�s la intensidad de la se�al es un poco baja y a un usuario bien entrenado requiere aproximadamente 30 segundos para escribir un caracter. % insertar imagen del articulo

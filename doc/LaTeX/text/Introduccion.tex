\chapter*{Introducci�n}
\markboth{\MakeUppercase{Introducci�n}}{}
\addcontentsline{toc}{chapter}{Introducci�n}

\indent El trabajo presentado describe la realizaci�n de un software libre desarrollado con ROS para ser utilizado con un lector de ondas cerebrales. El nuevo software tiene como objetivo la lectura, graficaci�n y registro de los datos que reciba a trav�s de la plataforma ROS. Durante la investigaci�n se encontr� que ya exist�a un software similar, sin embargo este resultaba muy lento lo que provocaba que la informaci�n en pantalla no fuera fiable. El software resultante en este trabajo tiene como objetivo resolver ese problema, es por eso que, como se explicar� en cap�tulos posteriores, se escogi� integrarlo a ROS.
\\ \indent El nuevo software permitir� controlar el momento en que se quiera Ejecutar, Pausar y Detener la recepci�n de la informaci�n, estas acciones no afectar�n al programa que env�a los datos pues solo afecta de forma visual al programa receptor para que el usuario tenga control de que quiere ver y en que momentos. Adicionalmente cuenta con la funcionalidad de visualizar y crear un registro de todas las se�ales que se reciban mientras el software se encuentre en modo de ``reproducci�n''. La visualizaci�n es e tiempo real mientras que la generaci�n del registro se genera cuando se presiona ``detener'' y crear� un archivo en disco duro con el nombre que se haya ingresado as� como la fecha y hora de creaci�n de este, dicho archivo permitir� al usuario consultarlo en cualquier momento independientemente de que el software se encuentre en ejecuci�n o no.
\section*{Antecedentes}
\markboth{\MakeUppercase{Antecedentes}}{}
\addcontentsline{toc}{section}{Antecedentes}

\indent El software desarrollado en este trabajo toma como base al emokit\cite{emokit} el cu�l forma parte de OpenYou. El Emokit lee, descifra e interpreta la informaci�n enviada por el Emotiv EPOC tales como nivel de bater�a de la diadema, intensidad de la se�al, y las 14 lecturas realizadas por la diadema; este software actualmente solo imprime a nivel terminal dichos datos. \\
\indent El Emotiv Epoc fue creado con el prop�sito de ser un perif�rico para juegos en Windows, OS X y Linux\cite{Emotiv:Wiki:Online}, cuenta con 14 electrodos y funciona como dispositivo de entrada. En 2011 Kirill Stytsenko, Evaldas Jablonskis y Cosima Prahm\cite{stytsenko2011evaluation} realizaron un an�lisis del Emotiv EPOC para la CogSci Conference. Emotiv EPOC es un BCI de bajo costo basado en la tecnolog�a EEG. Cuenta con 14 electrodos montados en una diadema inal�mbrica que se coloca sin esfuerzo y se conecta a la computadora. Originalmente fue creada para los juegos de computadora pero la ``research edition'' permite el acceso a los datos para su an�lisis lo que abre nuevas posibilidades a la ciencia para realizar nuevos experimentos o integrarlo a los ya existentes. En dicho estudio se someten a diferentes pruebas al Emotiv EPOC y al G-TEC\cite{G-TEC:Online}. Al compararlos se obtiene que la informaci�n en general es igual, pero la se�al es m�s clara e intensa en el G-TEC. Uno de los desafios encontrados es la creaci�n de software de grabaci�n para ambos dispositivos. \\
\indent La Universidad Veracruzana adquri� un Emotiv EPOC en 2014 para el laboratorio de rob�tica donde los alumnos de la Facultad de Ingenier�a participan en diferentes proyectos tecnol�gicos. Este trabajo de tesis contribuye a un proyecto donde se controlar� un robot por medio de la mente a trav�s del Emotiv EPOC. Con el proyecto resultante de este trabajo se tendr� una interfaz la cual graficar� las se�ales del Emotiv EPOC adem�s de tener la capacidad de grabar los datos obtenidos cada vez que se ejecute un ``experimento'' obteniendo un archivo Excel.

\begin{comment}
\indent En Interfaz Cerebro - Computadora para el Control de un Cursor Basado en Ondas Cerebrales\cite{ref2} se plantea una interfaz que permita la comunicaci�n entre el usuario y la computadora, haciendo uso de sus ondas cerebrales, para el control de un cursor en pantalla mediante comandos obtenidos de las lecturas de un amplificador de ondas cerebrales.
\\ \indent En EPOC-alypse Mind Controlled Car\cite{seniorepoc} plantean la construcci�n de de un carro de control remoto que es controlado por la mente usando el Emotiv EPOC. El proyecto fue desarrollado utilizando el SDK oficial del Emotiv EPOC.
\\ \indent En SSVEP based EEG Interface for Google Street View Navigation\cite{raza2012ssvep} analizan los sistemas BCI y su aplicaci�n en el mundo real. Tambi�n desarrollan  un prototipo interactivo que pueda ser controlado en un ambiente controlado para demostrar el funcionamiento de los sistemas BCI. Para el desarrollo decidieron utilizar el software libre OpenViBE para la adquisici�n y procesamiento de las se�ales.
\\ \indent En ROS: an open-source Robot Operating System\cite{2009ros} se explica el uso de la plataforma ROS para el desarrollo de aplicaciones de rob�tica. ROS provee una capa de comunicaci�n estructura basada en Peer-to-peer, basado en herramientas, adem�s es multilenguaje, ligero, gratuito y de c�digo abierto.
\\ \indent En Things that twitter: social networks and the internet of things\cite{kranz2010things} utilizan ROS aplicado en las redes sociales. ROS permite intercambiar informaci�n por medio de servicios con mensaje de request y response definidos. La informaci�n es intercambiada por una arquitectura publish/suscribe donde los procesos permiten que sus datos est�n disponibles para que otros procesos puedan utilizarlos.
\end{comment}
% DEscribir emokit, describir EPOC relacion con la UV, ROS,  trabajos previos 
\chapter*{Introducci�n}
\markboth{\MakeUppercase{Introducci�n}}{}
\addcontentsline{toc}{chapter}{Introducci�n}

Desde su surgimiento, el hombre ha tenido una capacidad de pensamiento superior a las otras especies, esta capacidad le ha permitido evolucionar a lo que somos actualmente. El pensamiento es la capacidad de realizar actividades como solucionar problemas, razonar, juzgar, categorizar, decidir o inventar\cite{ref1}, algunos investigadores se�alan la necesidad de tres ideas b�sicas para construir una definici�n general de pensamiento\cite{ref1}:
\begin{itemize}
\item ``El pensamiento es cognitivo pero se infiere de la conducta. Ocurre internamente, en la mente o el sistema cognitivo, y debe ser inferido indirectamente''.
\item ``El pensamiento es un proceso que implica alguna manipulaci�n de, o establece un conjunto de operaciones sobre, el conocimiento en el sistema cognitivo''.
\item ``El pensamiento es dirigido y tiene como resultado la resoluci�n de problemas, o se dirige hacia una soluci�n''.
\end{itemize}
Actualmente existen diferentes formas de leer y analizar como el ser humano piensa, en este trabajo se hablar� de  la lectura no invasiva de las ondas electroencefalogr�ficas, que son emitidas constantemente por el cerebro como resultado de la actividad cerebral\cite{ref2}.  Las lecturas no invasivas se realizan por medio de amplificadores, que obtienen la se�al a trav�s de electrodos conectados al cuero cabelludo. Estos amplificadores son programas que cuentan con filtros y t�cnicas de procesamiento para identificar las ondas emitidas\cite{ref2}. Sin embargo se encuentran limitados a las funciones que los creadores de dichos programas determinen.\\ \indent
Es por esto que la existencia de un software libre y de c�digo abierto es importante, pues permitir� a los desarrolladores hacer un uso m�s espec�fico de acuerdo a sus necesidades, y adem�s, con la posibilidad de adaptarlo y complementarlo. El presente trabajo plantea la creaci�n de dicho software libre con la finalidad de resolver las limitantes actuales o incluso, complementar los experimentos realizados con los programas que actualmente existen.
		
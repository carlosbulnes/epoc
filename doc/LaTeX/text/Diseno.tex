\chapter*{Dise�o}
\markboth{\MakeUppercase{Dise�o}}{}
\addcontentsline{toc}{chapter}{Dise�o}

\indent La Universidad Veracruzana adquri� un Emotiv EPOC en 2014 para el laboratorio de rob�tica donde los alumnos de la Facultad de Ingenier�a participan en diferentes proyectos tecnol�gicos. Este trabajo de tesis contribuye a un proyecto donde se controlar� un robot por medio de la mente a trav�s del Emotiv EPOC. Con el proyecto resultante de este trabajo se tendr� una interfaz la cual graficar� las se�ales del Emotiv EPOC adem�s de tener la capacidad de grabar los datos obtenidos cada vez que se ejecute un ``experimento'' obteniendo un archivo separado por comas.
\\ \indent Al momento de la investigaci�n se encontraron algunos software que mostraban la informaci�n obtenida de forma gr�fica, sin embargo el principal inconveniente de estos era el desfase entre los datos que se estaban obteniendo y su graficaci�n. Para atacar ese problema en este proyecto se utiliz� un framework llamado ROS. ROS (Robot Operating System)\cite{ROS:Wiki:Online} provee librer�as y herramientas para ayudar a los desarrolladores de software a crear aplicaciones para robots. ROS provee abstracci�n de hardware, controladores de dispositivos, librer�as, herramientas de visualizaci�n, comunicaci�n por mensajes, administraci�n de paquetes y m�s. ROS est� bajo la licencia open source, BSD. En este proyecto se utiliz� ROS para crear un escenario de comunicaci�n en tiempo real, donde el Emokit\cite{emokit} publica a trav�s de mensajes los datos obtenidos del Epoc as� mismo el nuevo software leer� los mensajes publicados en ROS para obtener los datos para graficarlos y generar el ``log'', todo esto ya en tiempo real.
\\ \indent Para la construcci�n del software presentado en este trabajo se utiliz�:
\begin{itemize}
\item Python\footnote{ Lenguaje de programaci�n que puede ser usado para diferentes prop�sitos como aplicaciones web, aplicaciones de escritorio, int�rprete, etc}
\item PyQT\footnote{  Binding de la biblioteca gr�fica Qt para el lenguaje de programaci�n Python}
\item Matplotlib\footnote{ Biblioteca para la generaci�n de gr�ficos a partir de datos contenidos en listas o arrays en el lenguaje de programaci�n Python y su extensi�n matem�tica NumPy}
\item ROS
\end{itemize}

\chapter*{Implementaci�n y Resultados}
\markboth{\MakeUppercase{Implementaci�n y Resultados}}{}
\addcontentsline{toc}{chapter}{Implementaci�n y Resultados}

La impelmentacion es una explicacion del software funcionado e imagenes de esllo
%Capturas y fotos de la aplicacion funcionando e interactuando con la diadema
\\ \indent A continuaci�n se presentan los c�digos con los que fue posible este proyecto, el proyecto completo puede ser descargado en \url{http://github.com/carlosbulnes/epoc/}.

Primero tenemos al c�digo del emokit adaptado a ROS.
\lstinputlisting[language=Python, caption=emokit.py]{code/emokit.py}

El c�digo del software realizado en este trabajo es el siguiente. Este est� constituido por un programa principal llamado interfaz.py el cu�l depende de los c�digos GUI.py que es la definici�n de la interfaz en PyQT y matplotlibwidgetFile.py que permite la integraci�n de matplotlib a PyQt.
\\ 
\lstinputlisting[language=Python, caption=interfaz.py]{code/interfaz.py}
\lstinputlisting[language=Python, caption=GUI.py]{code/GUI.py}
\lstinputlisting[language=Python, caption=matplotlibwidgetFile.py]{code/matplotlibwidgetFile.py}
%C�digos y su desarrollo
%Versiones

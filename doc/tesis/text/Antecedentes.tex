\chapter*{Antecedentes}
\markboth{\MakeUppercase{Antecedentes}}{}
\addcontentsline{toc}{chapter}{Antecedentes}

\indent En Interfaz Cerebro - Computadora para el Control de un Cursor Basado en Ondas Cerebrales\cite{ref2} se plantea una interfaz que permita la comunicaci�n entre el usuario y la computadora, haciendo uso de sus ondas cerebrales, para el control de un cursor en pantalla mediante comandos obtenidos de las lecturas de un amplificador de ondas cerebrales.
\\ \indent En EPOC-alypse Mind Controlled Car\cite{seniorepoc} plantean la construcci�n de de un carro de control remoto que es controlado por la mente usando el Emotiv EPOC. El proyecto fue desarrollado utilizando el SDK oficial del Emotiv EPOC.
\\ \indent En SSVEP based EEG Interface for Google Street View Navigation\cite{raza2012ssvep} analizan los sistemas BCI y su aplicaci�n en el mundo real. Tambi�n desarrollan  un prototipo interactivo que pueda ser controlado en un ambiente controlado para demostrar el funcionamiento de los sistemas BCI. Para el desarrollo decidieron utilizar el software libre OpenViBE para la adquisici�n y procesamiento de las se�ales.
\\ \indent En ROS: an open-source Robot Operating System\cite{2009ros} se explica el uso de la plataforma ROS para el desarrollo de aplicaciones de rob�tica. ROS provee una capa de comunicaci�n estructura basada en Peer-to-peer, basado en herramientas, adem�s es multilenguaje, ligero, gratuito y de c�digo abierto.
\\ \indent En Things that twitter: social networks and the internet of things\cite{kranz2010things} utilizan ROS aplicado en las redes sociales. ROS permite intercambiar informaci�n por medio de servicios con mensaje de request y response definidos. La informaci�n es intercambiada por una arquitectura publish/suscribe donde los procesos permiten que sus datos est�n disponibles para que otros procesos puedan utilizarlos.

% DEscribir emokit, describir EPOC relacion con la UV, ROS,  trabajos previos 